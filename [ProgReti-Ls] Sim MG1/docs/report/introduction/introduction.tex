\chapter{Introduzione}
\label{ch:intro}

L'esposizione delle attivit\`a svolte \`e stata organizzata, come suggerito all'interno del documento di rifermento, in cinque macro-sezioni all'interno delle quali sono poi stati sviluppati gli argomenti specifici.
In particolare, dopo la trattazione dei generatori di numeri pseudo-casuali analizzati si passer\`a allo studio delle varie tipologie di simulazioni svolte, concludendo con l'esposizione delle attivit\`a opzionali che sono state scelte.

\subsubsection{Struttura del report}

L'implementazione del simulatore rende necessaria l'esposizione del lavoro svolto secondo un duplice punto di vista: il primo relativo alla valutazione quantitativa e qualitativa degli esiti delle simulazioni svolte, il secondo (non meno importante) relativo all'infrastruttura tecnica implementata a garanzia della validit\`a dei dati presentati. Per questo motivo, all'interno di ciascuna sezione delineata nella struttura del report sopra indicata, si \`e cercato di fornire una trattazione esaustiva di entrambi gli aspetti. In particolare, in seguito all'esposizione delle simulazioni svolte, sar\`a posta una sezione di \emph{analisi tecnica} all'interno della quale verranno esposte le particolarit\`a implementative dei componenti di volta in volta coinvolti.
\\

La piattaforma di riferimento per lo sviluppo del simulatore \`e \emph{Java}, attraverso la quale sono stati sviluppati sia il simulatore in s\'e, sia la parte relativa alla costruzine dei grafici relativi alle simulazioni svolte.

\newpage
\subsubsection{Gruppo di lavoro}

Il gruppo di lavoro \`e costituito da:
\begin{itemize}
\item Michael Gattavecchia - \emph{0000362269}	
\item Marco Santarelli - \emph{0000366521}	
\item Andrea Zagnoli - \emph{0000367565}
\end{itemize}
