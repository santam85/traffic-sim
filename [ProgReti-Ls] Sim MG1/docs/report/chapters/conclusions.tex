\chapter{Conclusioni}

\section{Simulatore di teletraffico}

L'utilizzo di un simulatore di teletraffico ha dato la possibilit\`a di andare a valutare il comportamento di sistemi di servizio noti a fronte di richieste inoltrate da popolazioni generate artificialmente ma caratterizzate da comportamenti riconducibili a casi reali. L'aderenza tra i modelli di traffico generati ed i corrispettivi teorici \`e garantita dalle specifiche dei generatori di numeri pseudo-casuali ai quali si \`e fatto ricorso.
\\

In termini generali \`e comunque bene tenere conto del fatto che la simulazione di teletraffico rappresenti in s\'e un problema di grande complessit\`a. Al di l\`a delle considerazioni sulla qualit\`a dei numeri pseudo-casuali ai quali si fa riferimento \`e infatti importante tenere conto della realt\`a dei sistemi che si intendono simulare, nei quali le dinamiche di funzionamento sono determinate da grandi quantit\`a di variabili, spesso difficili anche da identificare, oltre che da esprimere attraverso un modello. 
\\

Fatta questa precisazione, la creazione e l'utilizzo di un simulatore possono sicuramente rappresentare una risorsa di indiscutibile utilit\`a per la sintesi ed il dimensionamento di sistemi di teletraffico per applicazioni reali, rendendo possibile un'immediata rappresentazione e valutazione di caratteristiche e dinamiche altrimenti difficili da contemplare.
